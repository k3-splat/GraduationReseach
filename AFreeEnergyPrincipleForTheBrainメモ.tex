\documentclass[a4paper]{jsarticle}

\usepackage[utf8]{inputenc}

\usepackage[dvipdfmx]{graphicx}
\usepackage[dvipdfmx]{xcolor}

\usepackage{hyperref}
\usepackage{mathtools}
\usepackage{amsmath}
\usepackage{amssymb}
\usepackage{amsfonts}
\usepackage{latexsym}
\usepackage{enumitem}
\usepackage{empheq}
\usepackage{amsthm}
\usepackage{bm}
\usepackage{physics}

\title{\Huge A free energy principle for the brain}
\author{B4 砂川}

\begin{document}

\maketitle

\section{abstract}
\begin{itemize}
    \item 自由エネルギー原理はヘルムホルツのアイデアを現代理論で定式化することによって組み上げられた.
    \item 統計力学の構造を用いることで脳の学習問題を(この論文内では)解決できるそう.
    \item 自由エネルギー原理は,経験的ベイズ法と,感覚入力がどのように発生するかの階層的モデルに基づいている.
    \item 階層的モデルの使用によって,脳は動的かつ文脈依存な方法で事前の予想を構築できる.
    \item 自由エネルギー原理によって,皮質組織の様々な側面と,その応答に関する原則を提供できる.
    \item これらの知覚プロセスはあくまで自由エネルギー原理に従う創発的な振舞いの一部に過ぎないそう.
    \item 自由エネルギーは,系に作用する環境量の確率分布と、系の構成によって符号化された任意の分布との差の指標と考えられる.
    \item 系はその構成を変化させることで,環境量のサンプリング方法に影響を与えたり,符号化する分布を変化させたりすることで,自由エネルギーを最小化することができる.
    \item この自由エネルギーの扱いは,系の状態と構造が環境の暗黙的かつ確率的なモデルを符号化していると仮定する.
\end{itemize}

\section{Introduction}
\begin{itemize}
    \item 概念モデルと数学モデルを構築するプロセスのモデル化を行う.
    \item 人間が世界について行う推論は,概念・数学モデルの推論のプロセスに適応できる.
    \item 階層的側面によって脳の事前学習は可能となり,暗黙の内に間隔データを生成する固有の因果構造を学習する.
    \item 知覚推論と学習は,どちらも自由エネルギーの最小化または予測誤差の抑制に基づいている.
    \item 自由エネルギーの概念は統計力学に由来し,機械学習において推論に内在する困難な積分問題をより容易な最適化問題に変換する.
    \item 自由エネルギーの最小化は,原理的には比較的単純なニューロン基盤を用いて実装可能.
\end{itemize}

\end{document}