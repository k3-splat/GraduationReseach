\documentclass[a4paper, titlepage]{jsarticle}

\usepackage[utf8]{inputenc}

\usepackage[dvipdfmx]{graphicx}
\usepackage[dvipdfmx]{xcolor}

\usepackage{hyperref}
\usepackage{mathtools}
\usepackage{amsmath}
\usepackage{amssymb}
\usepackage{amsfonts}
\usepackage{latexsym}
\usepackage{enumitem}
\usepackage{empheq}
\usepackage{amsthm}
\usepackage{bm}
\usepackage{physics}

\title{\Huge 自由エネルギー原理の解説\;:\;知覚・行動・他者の思考の推論}
\author{\Large 創域理工学部\quad 情報計算科学科\quad 3年\\\Large 学籍番号\;:\;6322045\\\Large 砂川恵太朗}
\date{提出日\;:\;\today}

\begin{document}

\maketitle

知覚の目的は
\begin{equation*}
    E_{p\qty(\tilde{s})}\qty[-\log p\qty(\tilde{s}|m)+\log p\qty(\tilde{s})]
\end{equation*}
を最小化することだそう.ここで\;$E_{p\qty(\tilde{s})}\qty[\cdot]$\;は$p\qty(\tilde{s})$についての期待値(アンサンブル平均)である.

\end{document}